\section{Herleitung ML-Sch�tzer}
Herleitung der ML-Sch�tzer f�r das BLIM, IMBLIM und MissBLIM Modell.

\noindent
F�r den M-Schritt des EM-Algorithmus ist eine ML-Sch�tzung der Parameter durchzuf�hren. Nachfolgend sollen die expliziten Sch�tzgleichungen f�r das IMBLIM und MissBLIM im Durchgang\footnote{Im Anhang wird auf eine Kennzeichnung des Durchgangs mit hochgestelltem $t$ verzichtet, alle Sch�tzungen beziehen sich jedoch auf den Durchgang $t$.} $t$ hergeleitet werden. $F(R, M, K)$ bezieht sich dabei auf die im E-Schritt berechneten erwarteten H�ufigkeiten $\mathcal{E}(F(R, M, K))$ (vgl. Gleichung \eqref{eq:E_FRMK}). Die Parametersch�tzungen des BLIM entsprechen aufgrund der Beziehung zwischen BLIM und IMBLIM denen des IMBLIM mit $M = \emptyset$.
\subsection*{IMBLIM}
Ausgehend von der Likelihood der vollst�ndigen Daten \eqref{eq:LH_IMBLIM} wird die log-Likelihood gebildet:
\begin{align*}
\log \mathcal{L}(\mathcal{Q}_K \mid \beta, \eta, \pi) = {}&\sum_{\langle R, M, K \rangle \in \mathcal{Q}_K}   F(R, M, K) \cdot \log \big( P(M) \cdot P(\langle R, M \rangle^{*} \mid K, \beta, \eta) \big) \\
												   & +  \sum_{\langle R, M, K \rangle \in \mathcal{Q}_K}   F(R, M, K) \cdot \log \pi_K \\
												   = {}&\sum_{\langle R, M, K \rangle \in \mathcal{Q}_K}   F(R, M, K) \cdot \log P(M) \\
												    & + \sum_{\langle R, M, K \rangle \in \mathcal{Q}_K}   F(R, M, K) \cdot \log  P(\langle R, M \rangle^{*} \mid K, \beta, \eta) \\
												    & + \sum_{\langle R, M, K \rangle \in \mathcal{Q}_K}   F(R, M, K) \cdot \log \pi_K
\end{align*}
\noindent
$\mathcal{Q}_K$ ist dabei die Menge aller theoretisch m�glichen Tripel $\langle R, M, K \rangle$. $\mathcal{Q}$ bezeichne weiter die Menge aller beobachtbaren Paare $\langle R, M \rangle$.

F�r die ML-Sch�tzung muss diese log-Likelihood nun partiell nach den Parametern abgeleitet und 0 gesetzt werden. Hierzu ist hilfreich, dass  
$$\sum_{\langle R, M, K \rangle \in \mathcal{Q}_K}  F(R, M, K) \cdot \log  P(\langle R, M \rangle^{*} \mid K)$$ nur von $\beta$ und $\eta$ abh�ngt und $$\sum_{\langle R, M, K \rangle \in \mathcal{Q}_K}    F(R, M, K) \cdot \log\pi_K$$ nur von $\pi_K$. Somit fallen bei der partiellen Ableitung die jeweils anderen Summanden weg, da sie von dem abzuleitenden Parameter unabh�ngig sind.

\subsubsection*{$\pi$-Parameter}
Hier ist der Teil $\log \mathcal{L}_{\pi} = \sum_{\langle R, M, K \rangle \in \mathcal{Q}_K}   F(R, M, K) \cdot \log \pi_K$  unter der Nebenbedingung $\sum_{K \in \mathcal{K}} \pi_K \overset{!}{=} 1$ zu maximieren. Dies kann durch die Einf�hrung eines Lagrange Multiplikators $\lambda$ erreicht werden:

\begin{align*}
\log \mathcal{L}_{\pi} = {}&\sum_{\langle R, M, K \rangle \in \mathcal{Q}_K}   F(R, M, K) \cdot \log \pi_K\\
\log \mathcal{L}_{\pi}^{Lagrange} = {}&  \sum_{\langle R, M, K \rangle \in \mathcal{Q}_K}    F(R, M, K) \cdot \log \pi_K  + \lambda \cdot (1 - \sum_{K \in \mathcal{K}} \pi_K)\\
\frac{\partial \log \mathcal{L}_{\pi}^{Lagrange}}{\partial \pi_K} = {}& \sum_{\langle R, M \rangle \in \mathcal{Q}} \frac{F(R, M, K)}{\pi_K} - \lambda \\
\intertext{Nullsetzten und aufl�sen:} 
\hat{\pi}_K = {}& \frac{1}{\lambda} \sum_{\langle R, M \rangle \in \mathcal{Q}} F(R, M, K)\\
\intertext{Wegen} 
1 = \sum_{K \in \mathcal{K}} \hat{\pi}_K = {}& \sum_{K \in \mathcal{K}}{ \frac{\sum_{\langle R, M \rangle \in \mathcal{Q}} F(R, M, K)}{\lambda}} = \frac{N}{\lambda} \\
\intertext{erh�lt man dann}
\hat{\pi}_K = {}& \frac{1}{N} \sum_{\langle R, M \rangle \in \mathcal{Q}} F(R, M, K)
\end{align*}
\noindent
wobei $N = \sum_{\langle R, M \rangle \in \mathcal{Q}}  \sum_{K \in \mathcal{K}}  F(R, M, K) = \sum_{\langle R, M, K \rangle \in \mathcal{Q}_K}  F(R, M, K) $ der Stichprobengr��e entspricht.

\subsubsection*{$\beta$-Parameter}
Hierzu ist der von $\beta$ abh�ngige Teil der log-Likelihood $\log \mathcal{L}_{\beta}$ zu maximieren mit der bedingten Wahrscheinlichkeit $P(\langle R, M \rangle^{*} \mid K, \beta, \eta)$ wie in Gleichung \eqref{eq:PRMS.K} definiert.
\begin{align*}
\log \mathcal{L}_{\beta} = {}& \sum_{\langle R, M, K \rangle \in \mathcal{Q}_K}    F(R, M, K) \cdot \log  P(\langle R, M \rangle^{*} \mid K, \beta, \eta)  \\
						 = {}& \sum_{\langle R, M, K \rangle \in \mathcal{Q}_K}    F(R, M, K) \cdot  \sum_{q \in K \setminus (R \cup M)} \log\beta_q  \\
						   & + \sum_{\langle R, M, K \rangle \in \mathcal{Q}_K}    F(R, M, K) \cdot  \sum_{q \in (K \cap R) \setminus M} \log(1 - \beta_q)  \\
						   & + \sum_{\langle R, M, K \rangle \in \mathcal{Q}_K}    F(R, M, K) \cdot \sum_{q \in R \setminus ( K \cup M)} \log\eta_q  \\
						   & + \sum_{\langle R, M, K \rangle \in \mathcal{Q}_K}    F(R, M, K) \cdot \sum_{q \in Q \setminus ( R \cup M \cup K)} \log(1-\eta_q)  \\
\frac{\partial \log \mathcal{L}_{\beta}}{\partial \beta_q} = {}&  \sum_{\langle R, M, K \rangle \in \mathcal{Q}_K}  F(R, M, K) \cdot  \sum_{q \in K \setminus (R \cup M)} \frac{1}{\beta_q} \\
						   & + \sum_{\langle R, M, K \rangle \in \mathcal{Q}_K}    F(R, M, K) \cdot  \sum_{q \in (K \cap R) \setminus M} \frac{1}{\beta_q - 1}  
\end{align*}
\noindent
Einf�hrung von zwei Indikatorvariablen nach dem Schema wie in Gleichung \eqref{eq:indikatorvariable}:
\begin{align*} 
i_{q \in K \setminus (R \cup M)} = {}&\begin{cases}
1 & \text{,wenn}\quad q \in K \setminus (R \cup M)\\
0 & \text{sonst}        
\end{cases}\\
i_{q \in (K \cap R) \setminus M} = {}& \begin{cases}
1 & \text{,wenn}\quad q \in (K \cap R) \setminus M\\
0 & \text{sonst}        
\end{cases} 
\end{align*}
\noindent
Daraus folgt:
\begin{align*}  
\frac{\partial \log \mathcal{L}_{\beta}}{\partial \beta_q} = {}&  \sum_{\langle R, M, K \rangle \in \mathcal{Q}_K} F(R, M, K) \cdot i_{q \in K \setminus (R \cup M)} \cdot \frac{1}{\beta_q}  \\
						                                      & + \sum_{\langle R, M, K \rangle \in \mathcal{Q}_K}  F(R, M, K) \cdot  i_{q \in (K \cap R) \setminus M} \cdot \frac{1}{\beta_q - 1} 
\end{align*}
\noindent
Nullsetzen:
\begin{align*} 
0 = {}& \sum_{\langle R, M, K \rangle \in \mathcal{Q}_K} F(R, M, K) \cdot i_{q \in K \setminus (R \cup M)} \cdot \frac{1}{\hat{\beta}_q} \\
	& + \sum_{\langle R, M, K \rangle \in \mathcal{Q}_K}  F(R, M, K) \cdot  i_{q \in (K \cap R) \setminus M} \cdot \frac{1}{\hat{\beta}_q - 1} \\
0 = {}& (\hat{\beta}_q - 1) \sum_{\langle R, M, K \rangle \in \mathcal{Q}_K}  F(R, M, K) \cdot i_{q \in K \setminus (R \cup M)} \\
    & + \hat{\beta}_q \sum_{\langle R, M, K \rangle \in \mathcal{Q}_K}   F(R, M, K) \cdot  i_{q \in (K \cap R) \setminus M}\\
\hat{\beta}_q = {}& \frac{\sum_{\langle R, M, K \rangle \in \mathcal{Q}_K}  F(R, M, K) \cdot i_{q \in K \setminus (R \cup M)}}{\sum_{\langle R, M, K \rangle \in \mathcal{Q}_K}   F(R, M, K) \cdot i_{q \in K \setminus (R \cup M)} + \sum_{\langle R, M, K \rangle \in \mathcal{Q}_K}  F(R, M, K) \cdot  i_{q \in (K \cap R) \setminus M}}\\
= {}& \frac{\sum_{\langle R, M, K \rangle \in \mathcal{Q}_K}   F(R, M, K) \cdot i_{q \in K \setminus (R \cup M)}}{\sum_{\langle R, M, K \rangle \in \mathcal{Q}_K}  F(R, M, K) \cdot \big(i_{q \in K \setminus (R \cup M)} +  i_{q \in (K \cap R) \setminus M}\big)}\\
= {}& \frac{\sum_{\langle R, M, K \rangle \in \mathcal{Q}_K}   F(R, M, K) \cdot i_{q \in K \setminus (R \cup M)}}{\sum_{\langle R, M, K \rangle \in \mathcal{Q}_K}  F(R, M, K) \cdot i_{q \in K \setminus M}}
\end{align*}

\subsubsection*{$\eta$-Parameter}
Das Vorgehen ist aufgrund der sehr �hnlichen Struktur der Gleichung \eqref{eq:PRMS.K} f�r $\beta$ und $\eta$ �quivalent und f�hrt zu folgender Sch�tzgleichung mit angepassten Indikatorvariablen $i_{q \in R \setminus ( K \cup M)}$ und $i_{q \in Q \setminus ( R \cup M \cup K)}$, welche sich aufgrund $R \cap M = \emptyset$ vereinfachen lassen:
\begin{align*}
\hat{\eta}_q = {}&\frac{\sum_{\langle R, M, K \rangle \in \mathcal{Q}_K}  F(R, M, K) \cdot i_{q \in R \setminus ( K \cup M)}}{\sum_{\langle R, M, K \rangle \in \mathcal{Q}_K} F(R, M, K) \cdot \big(i_{q \in R \setminus ( K \cup M)} +  i_{q \in Q \setminus ( R \cup M \cup K)} \big)}\\
= {}&\frac{\sum_{\langle R, M, K \rangle \in \mathcal{Q}_K}  F(R, M, K) \cdot i_{q \in R \setminus ( K \cup M)}}{\sum_{\langle R, M, K \rangle \in \mathcal{Q}_K}  F(R, M, K) \cdot i_{q \in Q \setminus (M \cup K)}}
\end{align*}

\clearpage

\subsection*{MissBLIM}
Wieder ausgehend von der Likelihood der vollst�ndigen Daten \eqref{eq:LH_MissBLIM} diesmal des MissBLIM wird die log-Likelihood gebildet:

\begin{align*}
\log \mathcal{L}(\mathcal{Q}_K | \beta, \eta, \pi, \mu, \bar{\mu}) = {}&\sum_{\langle R, M, K \rangle \in \mathcal{Q}_K}   F(R, M, K) \cdot \log \big(P(\langle R, M \rangle^{*} \mid K, \beta, \eta)\\
												& \cdot P(M\mid K, \mu, \bar{\mu})\big) \\
												& +  \sum_{\langle R, M, K \rangle \in \mathcal{Q}_K}    F(R, M, K) \cdot \log \pi_K \\
												= {}&\sum_{\langle R, M, K \rangle \in \mathcal{Q}_K}  F(R, M, K) \cdot \log  P(\langle R, M \rangle^{*} \mid K, \beta, \eta) \\
												& +\sum_{\langle R, M, K \rangle \in \mathcal{Q}_K}   F(R, M, K) \cdot \log P(M\mid K, \mu, \bar{\mu})  \\
												& + \sum_{\langle R, M, K \rangle \in \mathcal{Q}_K}    F(R, M, K) \cdot \log \pi_K
\end{align*}

Die Sch�tzer f�r $\beta$, $\eta$ und $\pi$ sind dabei �quivalent zu denen des IMBLIM, da der von ihnen abh�ngige Teil der Likelihood gleich ist. Hinzu kommen hier jedoch noch die
 $\mu = (\mu_q)_{q \in Q}$ und $\bar{\mu} = (\mu_{\bar{q}})_{q \in Q}$ Parameter: $\mu_q$ als die Wahrscheinlichkeit, dass ein Item $q$, das im Wissenszustand $K$ enthalten ist, nicht beantwortet wird (weder richtig, noch falsch) und $\mu_{\bar{q}}$, die Wahrscheinlichkeit, dass ein Item, das sich nicht im Wissenszustand der Person befindet, nicht beantwortet wird. F�r die ML-Sch�tzung dieser Parameter muss der von diesen abh�ngige Teil der Log-Likelihood maximiert werden.
 
\subsubsection*{$\mu$- und $\bar{\mu}$-Parameter}
Der von $\mu$ und $\bar{\mu}$ abh�ngige Teil der log-Likelihood lautet:

$$\log \mathcal{L}_{\mu} = \sum_{\langle R, M, K \rangle \in \mathcal{Q}_K}  F(R, M, K) \cdot \log P(M\mid K, \mu, \bar{\mu}) $$

\noindent
mit $P(M\mid K, \mu, \bar{\mu})$ wie in Gleichung \eqref{eq:PM.K} definiert. F�r das einfachere Ableiten kann diese Gleichung umgeschrieben werden \citep[vgl.][]{DeChiusole2015}:
$$ P(M\mid K, \mu, \bar{\mu}) = \left[ \prod_{q \in K \cap M} \mu_{q} \right ] \left[ \prod_{q \in K \setminus M} (1 - \mu_q) \right] \left[\prod_{q \in M \setminus K} \mu_{\bar{q}} \right] \left[\prod_{q \in Q \setminus (M \cup K)} (1 - \mu_{\bar{q}}) \right] $$

\noindent
Damit ergibt sich dann f�r die log-Likelihood und die Ableitung
\begin{align*}
\log \mathcal{L}_{\mu} = {}& \sum_{\langle R, M, K \rangle \in \mathcal{Q}_K}   F(R, M, K) \cdot \sum_{q \in K \cap M} \log \mu_q\\
						 & + \sum_{\langle R, M, K \rangle \in \mathcal{Q}_K}   F(R, M, K) \cdot \sum_{q \in K \setminus M} \log(1 - \mu_q)\\
						 & + \sum_{\langle R, M, K \rangle \in \mathcal{Q}_K}   F(R, M, K) \cdot \sum_{q \in M \setminus K} \log \mu_{\bar{q}}\\
						 & + \sum_{\langle R, M, K \rangle \in \mathcal{Q}_K}   F(R, M, K) \cdot \sum_{q \in Q \setminus (M \cup K)} \log (1 - \mu_{\bar{q}})\\
\frac{\partial \log \mathcal{L}_{\mu}} {\partial \mu_q}= {}& \sum_{\langle R, M, K \rangle \in \mathcal{Q}_K}   F(R, M, K) \cdot \sum_{q \in K \cap M} \frac{1} {\mu_q}\\
						 & + \sum_{\langle R, M, K \rangle \in \mathcal{Q}_K}  F(R, M, K) \cdot \sum_{q \in K \setminus M} \frac{1}{\mu_q - 1}
\end{align*}			

Das weitere Vorgehen ist dann sehr �hnlich zu der $\beta$- bzw. $\eta$-Sch�tzung des IMBLIM, da die Gleichungen strukturell �quivalent sind. Es werden wieder zwei Indikatorvariablen zur Hilfe genommen:
\begin{align*} 
i_{q \in K \cap M} = {}&\begin{cases}
1 & \text{,wenn}\quad q \in K \cap M\\
0 & \text{sonst}        
\end{cases}\\
i_{q \in K \setminus M} = {}& \begin{cases}
1 & \text{,wenn}\quad q \in K \setminus M\\
0 & \text{sonst}        
\end{cases} 
\end{align*}

\noindent
Damit ergibt sich dann die ML-Sch�tzung f�r $\mu_q$:
\begin{align*} 
\hat{\mu}_q = {}& \frac{\sum_{\langle R, M, K \rangle \in \mathcal{Q}_K}  F(R, M, K) \cdot i_{q \in K \cap M}}{\sum_{\langle R, M, K \rangle \in \mathcal{Q}_K}  F(R, M, K) \cdot \big( i_{q \in K \cap M} +  i_{q \in K \setminus M} \big)} \\
= {} & \frac{\sum_{\langle R, M, K \rangle \in \mathcal{Q}_K}   F(R, M, K) \cdot i_{q \in K \cap M}}{\sum_{\langle R, M, K \rangle \in \mathcal{Q}_K}  F(R, M, K) \cdot  i_{q \in K}}
\end{align*} 

und f�r $\mu_{\bar{q}}$:
\begin{align*} 
\hat{\mu}_{\bar{q}} = {}& \frac{\sum_{\langle R, M, K \rangle \in \mathcal{Q}_K}  F(R, M, K) \cdot i_{q \in M \setminus K}}{\sum_{\langle R, M, K \rangle \in \mathcal{Q}_K}  F(R, M, K) \cdot \big( i_{q \in M \setminus K} +  i_{q \in Q \setminus (M \cup K)} \big)} \\
= {} & \frac{\sum_{\langle R, M, K \rangle \in \mathcal{Q}_K}   F(R, M, K) \cdot i_{q \in M \setminus K}}{\sum_{\langle R, M, K \rangle \in \mathcal{Q}_K} F(R, M, K) \cdot  i_{q \in Q \setminus K}}
\end{align*}  


