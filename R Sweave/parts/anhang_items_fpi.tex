\section{Items FPI-R}
Die verwendeten Items des FPI-R f�r die Skalen lauten:

\begin{table}[hbtp]
	\caption[FPI-R Items der Skala Gesundheitssorgen]{12 Items der Skala Gesundheitssorgen.}
	\begin{tabular}{@{\extracolsep{\fill}} c l l @{}} 
		\toprule
		Skala & Item & Aussage                                                                 \\
		\midrule
		\parbox[t]{2mm}{\multirow{20}{*}{\rotatebox[origin=c]{90}{Gesundheitssorgen}}}
		      & 10   & Im Krankheitsfall m�chte ich Befund und Behandlung eigentlich von       \\                                            
		      &      & \,     einem zweiten Arzt �berpr�fen lassen.                            \\           
		      & 18   & Ich achte aus gesundheitlichen Gr�nden auf regelm��ige Mahlzeiten       \\                                            
		      &      & \,    und reichlichen Schlaf.                                           \\              
		      & 31   & Ich habe mich �ber die h�ufigsten Krankheiten und ihre ersten           \\                                            
		      &      & \,   Anzeichen informiert.                                              \\               
		      & 38   & Um gesund zu bleiben, achte ich auf ein ruhiges Leben.                  \\                                                       
		      & 62   & Ich vermeide es, ungewaschenes Obst zu essen.                           \\                                                             
		      & 65   & Ich vermeide Zugluft, weil man sich zu leicht erk�lten kann.            \\                                                    
		      & 68   & Wenn jemand in meine Richtung hustet oder niest, versuche ich mich      \\                                            
		      &      & \,  abzuwenden.                                                         \\                                      
		      & 70   & Ich hole sicherheitshalber �rztlichen Rat ein, wenn ich l�nger als      \\                                            
		      &      & \, zwei Tage erh�hte Temperatur (leichtes Fieber) habe.                 \\
		      & 84   & Weil man sich so leicht anstecken kann, wasche ich mir zu Hause         \\                                            
		      &      & \,  gleich die H�nde.                                                   \\                     
		      & 89   & Ich passe auf, dass ich nicht zu viel Autoabgase und Staub einatme.     \\                                            
		      & 117  & Handt�cher in viel benutzten Waschr�umen sind mir wegen der             \\                                            
		      &      & \,  Ansteckungsgefahr unangenehm.                                       \\                         
		      & 127  & Auch ohne ernste Beschwerden gehe ich regelm��ig zum Arzt, nur zur      \\                                            
		      &      & \,  Vorsicht.                                                           \\                                     						
		\bottomrule					
	\end{tabular}
	\smallskip
	\small\textit{Anmerkung}. Negativ gepolte Items sind mit (n) gekennzeichnet.
\end{table}

\begin{table}[htp]
    \caption[FPI-R Items der Skalen Erregbarkeit und Gehemmtheit]{Je 12 Items der Skalen Erregbarkeit und Gehemmtheit.}
	\begin{tabular}{@{\extracolsep{\fill}} c l l @{}} 
		\toprule
		Skala & Item & Aussage                                                                 \\
		\midrule
		\parbox[t]{2mm}{\multirow{15}{*}{\rotatebox[origin=c]{90}{Erregbarkeit}}}
		      & 27   & Ich neige dazu, bei Auseinandersetzungen lauter zu sprechen als sonst.  \\                
		      & 30   & Wenn mir einmal etwas schiefgeht, regt mich das nicht weiter auf. (n)   \\                  
		      & 52   & Auch wenn es eher viel zu tun gibt, lasse ich mich nicht hetzen. (n)    \\                  
		      & 60   & Auch wenn mich etwas sehr aus der Fassung bringt, beruhige ich mich     \\
		      &      & \, meistens wieder rasch. (n)                                           \\
		      & 86   & Mein Blut kocht, wenn man mich zum Narren h�lt.                         \\                     
		      & 93   & Es gibt nur wenige Dinge, die mich leicht erregen oder �rgern. (n)      \\                       
		      & 102  & Im allgemeinen bin ich ruhig und nicht leicht aufzuregen. (n)           \\                     
		      & 105  & Ich kann oft meinen �rger und meine Wut nicht beherrschen.              \\                
		      & 108  & Ich lasse mich durch eine Vielzahl von kleinen St�rungen nicht aus      \\
		      &      & \, der Ruhe bringen. (n)                                                \\
		      & 113  & Ich neige oft zu Hast und Eile, auch wenn es �berhaupt nicht notwendig  \\  
		      &      & \, ist.                                                                 \\
		      & 115  & Oft rege ich mich zu rasch �ber jemanden auf.                           \\     
		      & 135  & Ich bin leicht aus der Ruhe gebracht, wenn ich angegriffen werde.       \\
		\midrule
		\parbox[t]{2mm}{\multirow{18}{*}{\rotatebox[origin=c]{90}{Gehemmtheit}}}
		      & 4    & Ich habe fast immer eine schlagfertige Antwort bereit. (n)              \\                                                        
		      & 6    & Ich scheue mich, allein in einen Raum zu gehen, in dem andere Leute     \\
		      &      & \, bereits zusammensitzen und sich unterhalten.                         \\
		      & 8    & Ich w�rde mich beim Kellner oder Gesch�ftsf�hrer eines Restaurants      \\
		      &      & \, beschweren, wenn ein schlechtes Essen serviert wird. (n)             \\
		      & 11   & Ich bin ungern mit Menschen zusammen, die ich noch nicht kenne.         \\                                                                
		      & 63   & Es f�llt mir schwer, vor einer gro�en Gruppe von Menschen zu            \\
		      &      & \, sprechen oder vorzutragen.                                           \\    
		      & 73   & Ich bin im Grunde eher ein �ngstlicher Mensch.                          \\                                                             
		      & 81   & Ich schlie�e nur langsam Freundschaften.                                \\                                                           
		      & 85   & Ich werde ziemlich leicht verlegen.                                     \\                                                          
		      & 97   & Ich err�te leicht.                                                      \\                                                       
		      & 109  & Bei Geselligkeiten und �ffentlichen Veranstaltungen bleibe ich lieber   \\
		      &      & \, im Hintergrund.                                                      \\ 
		      & 120  & Beim Reisen schaue ich lieber auf die Landschaft als mich mit den       \\
		      &      & \, Mitreisenden zu unterhalten.                                         \\
		      & 124  & Es f�llt mir schwer, den richtigen Gespr�chsstoff  zu finden, wenn      \\ 
		      &      & \, ich jemanden kennenlernen will.                                      \\								
		\bottomrule					
	\end{tabular}
	\smallskip
	\small\textit{Anmerkung}. Negativ gepolte Items sind mit (n) gekennzeichnet.
\end{table}